\documentclass[]{article}
\usepackage{lmodern}
\usepackage{amssymb,amsmath}
\usepackage{ifxetex,ifluatex}
\usepackage{fixltx2e} % provides \textsubscript
\ifnum 0\ifxetex 1\fi\ifluatex 1\fi=0 % if pdftex
  \usepackage[T1]{fontenc}
  \usepackage[utf8]{inputenc}
\else % if luatex or xelatex
  \ifxetex
    \usepackage{mathspec}
  \else
    \usepackage{fontspec}
  \fi
  \defaultfontfeatures{Ligatures=TeX,Scale=MatchLowercase}
\fi
% use upquote if available, for straight quotes in verbatim environments
\IfFileExists{upquote.sty}{\usepackage{upquote}}{}
% use microtype if available
\IfFileExists{microtype.sty}{%
\usepackage{microtype}
\UseMicrotypeSet[protrusion]{basicmath} % disable protrusion for tt fonts
}{}
\usepackage[margin=1in]{geometry}
\usepackage{hyperref}
\hypersetup{unicode=true,
            pdftitle={K-means Clustering},
            pdfauthor={Muhammad Apriandito},
            pdfborder={0 0 0},
            breaklinks=true}
\urlstyle{same}  % don't use monospace font for urls
\usepackage{color}
\usepackage{fancyvrb}
\newcommand{\VerbBar}{|}
\newcommand{\VERB}{\Verb[commandchars=\\\{\}]}
\DefineVerbatimEnvironment{Highlighting}{Verbatim}{commandchars=\\\{\}}
% Add ',fontsize=\small' for more characters per line
\usepackage{framed}
\definecolor{shadecolor}{RGB}{248,248,248}
\newenvironment{Shaded}{\begin{snugshade}}{\end{snugshade}}
\newcommand{\AlertTok}[1]{\textcolor[rgb]{0.94,0.16,0.16}{#1}}
\newcommand{\AnnotationTok}[1]{\textcolor[rgb]{0.56,0.35,0.01}{\textbf{\textit{#1}}}}
\newcommand{\AttributeTok}[1]{\textcolor[rgb]{0.77,0.63,0.00}{#1}}
\newcommand{\BaseNTok}[1]{\textcolor[rgb]{0.00,0.00,0.81}{#1}}
\newcommand{\BuiltInTok}[1]{#1}
\newcommand{\CharTok}[1]{\textcolor[rgb]{0.31,0.60,0.02}{#1}}
\newcommand{\CommentTok}[1]{\textcolor[rgb]{0.56,0.35,0.01}{\textit{#1}}}
\newcommand{\CommentVarTok}[1]{\textcolor[rgb]{0.56,0.35,0.01}{\textbf{\textit{#1}}}}
\newcommand{\ConstantTok}[1]{\textcolor[rgb]{0.00,0.00,0.00}{#1}}
\newcommand{\ControlFlowTok}[1]{\textcolor[rgb]{0.13,0.29,0.53}{\textbf{#1}}}
\newcommand{\DataTypeTok}[1]{\textcolor[rgb]{0.13,0.29,0.53}{#1}}
\newcommand{\DecValTok}[1]{\textcolor[rgb]{0.00,0.00,0.81}{#1}}
\newcommand{\DocumentationTok}[1]{\textcolor[rgb]{0.56,0.35,0.01}{\textbf{\textit{#1}}}}
\newcommand{\ErrorTok}[1]{\textcolor[rgb]{0.64,0.00,0.00}{\textbf{#1}}}
\newcommand{\ExtensionTok}[1]{#1}
\newcommand{\FloatTok}[1]{\textcolor[rgb]{0.00,0.00,0.81}{#1}}
\newcommand{\FunctionTok}[1]{\textcolor[rgb]{0.00,0.00,0.00}{#1}}
\newcommand{\ImportTok}[1]{#1}
\newcommand{\InformationTok}[1]{\textcolor[rgb]{0.56,0.35,0.01}{\textbf{\textit{#1}}}}
\newcommand{\KeywordTok}[1]{\textcolor[rgb]{0.13,0.29,0.53}{\textbf{#1}}}
\newcommand{\NormalTok}[1]{#1}
\newcommand{\OperatorTok}[1]{\textcolor[rgb]{0.81,0.36,0.00}{\textbf{#1}}}
\newcommand{\OtherTok}[1]{\textcolor[rgb]{0.56,0.35,0.01}{#1}}
\newcommand{\PreprocessorTok}[1]{\textcolor[rgb]{0.56,0.35,0.01}{\textit{#1}}}
\newcommand{\RegionMarkerTok}[1]{#1}
\newcommand{\SpecialCharTok}[1]{\textcolor[rgb]{0.00,0.00,0.00}{#1}}
\newcommand{\SpecialStringTok}[1]{\textcolor[rgb]{0.31,0.60,0.02}{#1}}
\newcommand{\StringTok}[1]{\textcolor[rgb]{0.31,0.60,0.02}{#1}}
\newcommand{\VariableTok}[1]{\textcolor[rgb]{0.00,0.00,0.00}{#1}}
\newcommand{\VerbatimStringTok}[1]{\textcolor[rgb]{0.31,0.60,0.02}{#1}}
\newcommand{\WarningTok}[1]{\textcolor[rgb]{0.56,0.35,0.01}{\textbf{\textit{#1}}}}
\usepackage{graphicx,grffile}
\makeatletter
\def\maxwidth{\ifdim\Gin@nat@width>\linewidth\linewidth\else\Gin@nat@width\fi}
\def\maxheight{\ifdim\Gin@nat@height>\textheight\textheight\else\Gin@nat@height\fi}
\makeatother
% Scale images if necessary, so that they will not overflow the page
% margins by default, and it is still possible to overwrite the defaults
% using explicit options in \includegraphics[width, height, ...]{}
\setkeys{Gin}{width=\maxwidth,height=\maxheight,keepaspectratio}
\IfFileExists{parskip.sty}{%
\usepackage{parskip}
}{% else
\setlength{\parindent}{0pt}
\setlength{\parskip}{6pt plus 2pt minus 1pt}
}
\setlength{\emergencystretch}{3em}  % prevent overfull lines
\providecommand{\tightlist}{%
  \setlength{\itemsep}{0pt}\setlength{\parskip}{0pt}}
\setcounter{secnumdepth}{0}
% Redefines (sub)paragraphs to behave more like sections
\ifx\paragraph\undefined\else
\let\oldparagraph\paragraph
\renewcommand{\paragraph}[1]{\oldparagraph{#1}\mbox{}}
\fi
\ifx\subparagraph\undefined\else
\let\oldsubparagraph\subparagraph
\renewcommand{\subparagraph}[1]{\oldsubparagraph{#1}\mbox{}}
\fi

%%% Use protect on footnotes to avoid problems with footnotes in titles
\let\rmarkdownfootnote\footnote%
\def\footnote{\protect\rmarkdownfootnote}

%%% Change title format to be more compact
\usepackage{titling}

% Create subtitle command for use in maketitle
\providecommand{\subtitle}[1]{
  \posttitle{
    \begin{center}\large#1\end{center}
    }
}

\setlength{\droptitle}{-2em}

  \title{K-means Clustering}
    \pretitle{\vspace{\droptitle}\centering\huge}
  \posttitle{\par}
    \author{Muhammad Apriandito}
    \preauthor{\centering\large\emph}
  \postauthor{\par}
      \predate{\centering\large\emph}
  \postdate{\par}
    \date{5/21/2019}


\begin{document}
\maketitle

\hypertarget{k-means-cluster-analysis}{%
\subsection{K-means Cluster Analysis}\label{k-means-cluster-analysis}}

Clustering adalah serangkaian teknik untuk menemukan sub-kelompok dalam
suatu set data. Clustering memungkinkan kita untuk mengidentifikasi
sub-kelompok yang sama, dan kemunkinan dapat dikategorikan.

K-means clustering adalah metode pengelompokan yang paling sederhana dan
umum digunakan untuk memisahkan dataset ke dalam sekumpulan sub-kelompok
(K).

\hypertarget{library}{%
\subsection{Library}\label{library}}

Untuk membuat K-Means Clusterering di R, ada beberapa package yang
dibutuhkan diantaranya package ``tidyverse'', ``cluster'', dan
``factoextra''.

tidyverse -\textgreater{} Data Manipulation cluster -\textgreater{}
Clustering Algorithms factoextra -\textgreater{} Clustering algorithms
\& visualization

\begin{Shaded}
\begin{Highlighting}[]
\CommentTok{# Panggil library yang diperlukan}
\KeywordTok{library}\NormalTok{(tidyverse)}
\end{Highlighting}
\end{Shaded}

\begin{verbatim}
## Registered S3 methods overwritten by 'ggplot2':
##   method         from 
##   [.quosures     rlang
##   c.quosures     rlang
##   print.quosures rlang
\end{verbatim}

\begin{verbatim}
## -- Attaching packages ------------------------------------------ tidyverse 1.2.1 --
\end{verbatim}

\begin{verbatim}
## v ggplot2 3.1.1     v purrr   0.3.2
## v tibble  2.1.1     v dplyr   0.8.1
## v tidyr   0.8.3     v stringr 1.4.0
## v readr   1.3.1     v forcats 0.4.0
\end{verbatim}

\begin{verbatim}
## -- Conflicts --------------------------------------------- tidyverse_conflicts() --
## x dplyr::filter() masks stats::filter()
## x dplyr::lag()    masks stats::lag()
\end{verbatim}

\begin{Shaded}
\begin{Highlighting}[]
\KeywordTok{library}\NormalTok{(cluster)}
\KeywordTok{library}\NormalTok{(factoextra)}
\end{Highlighting}
\end{Shaded}

\begin{verbatim}
## Welcome! Related Books: `Practical Guide To Cluster Analysis in R` at https://goo.gl/13EFCZ
\end{verbatim}

\hypertarget{data-preparation}{%
\subsection{Data Preparation}\label{data-preparation}}

Umumnya, dalam melakukan Cluster Analysis. Beberapa tahapan data
pre-procssing yang perlu dilakukan :

\begin{enumerate}
\def\labelenumi{\arabic{enumi}.}
\tightlist
\item
  Menghapus/Mengolah Data Hilang
\item
  Normalisasi/Normalisasi
\end{enumerate}

Dalam praktek kali ini kita akan menggunakan dataset bawaan R yaitu
``USArrests''. yang merupakan dataset menganai tingkat kejahatan di 50
Negara Bagian Amerika Serikat pada tahun 1973.

\begin{Shaded}
\begin{Highlighting}[]
\CommentTok{#Memanggil dataset}
\NormalTok{df <-}\StringTok{ }\NormalTok{USArrests}

\CommentTok{#Melihat Dimensi Data dim(), berapa kolom dan berapa variable}
\KeywordTok{dim}\NormalTok{(df)}
\end{Highlighting}
\end{Shaded}

\begin{verbatim}
## [1] 50  4
\end{verbatim}

\begin{Shaded}
\begin{Highlighting}[]
\CommentTok{#Melihat 10 Baris Pertama mnggunakan fungsi head}
\KeywordTok{head}\NormalTok{(df)}
\end{Highlighting}
\end{Shaded}

\begin{verbatim}
##            Murder Assault UrbanPop Rape
## Alabama      13.2     236       58 21.2
## Alaska       10.0     263       48 44.5
## Arizona       8.1     294       80 31.0
## Arkansas      8.8     190       50 19.5
## California    9.0     276       91 40.6
## Colorado      7.9     204       78 38.7
\end{verbatim}

\begin{Shaded}
\begin{Highlighting}[]
\CommentTok{#Mengecek apakah ada Data Kosong (N/A) dalam data. }
\KeywordTok{sum}\NormalTok{(}\KeywordTok{is.na}\NormalTok{(df))}
\end{Highlighting}
\end{Shaded}

\begin{verbatim}
## [1] 0
\end{verbatim}

{[}1{]} 0 Dapat dilihat bahwa tidak ada data kosong didalam dataset
tersebut, sehingga tidak perlu adanya langkah lebih lanjut terkait data
kosong.

Selanjutnya, untuk membuat semua feature dalam dataset memliki skala
yang sama (0-1) kita perlu melalukan standarisasi / penyetaraan skala.
Standarisasi dapat dilakukan menggunkan fungsi ``scale()''.

\begin{Shaded}
\begin{Highlighting}[]
\CommentTok{#Lakukan Standarisasi dan assign ke variable baru dengan nama "dfnorm".  }
\NormalTok{dfnorm <-}\StringTok{ }\KeywordTok{scale}\NormalTok{(df)}
\end{Highlighting}
\end{Shaded}

Mari kita cek data kita, setelah dilakukan normalisasi:

\begin{Shaded}
\begin{Highlighting}[]
\CommentTok{#Melihat 10 data teratas pada dataset yang telah dinormalisasi}
\KeywordTok{head}\NormalTok{(dfnorm, }\DecValTok{10}\NormalTok{)}
\end{Highlighting}
\end{Shaded}

\begin{verbatim}
##                  Murder    Assault   UrbanPop         Rape
## Alabama      1.24256408  0.7828393 -0.5209066 -0.003416473
## Alaska       0.50786248  1.1068225 -1.2117642  2.484202941
## Arizona      0.07163341  1.4788032  0.9989801  1.042878388
## Arkansas     0.23234938  0.2308680 -1.0735927 -0.184916602
## California   0.27826823  1.2628144  1.7589234  2.067820292
## Colorado     0.02571456  0.3988593  0.8608085  1.864967207
## Connecticut -1.03041900 -0.7290821  0.7917228 -1.081740768
## Delaware    -0.43347395  0.8068381  0.4462940 -0.579946294
## Florida      1.74767144  1.9707777  0.9989801  1.138966691
## Georgia      2.20685994  0.4828549 -0.3827351  0.487701523
\end{verbatim}

\hypertarget{clustering-distance-measures}{%
\subsection{Clustering Distance
Measures}\label{clustering-distance-measures}}

Pemilihan Disctance Measures merupakan tahapan penting dalam clustering,
karena akan berpengaruh pada hasil/bentuk clustering. Beberapa distance
measures yang umum digunakan adalah Euclidean and Manhattan
distances.Pada praktek kali ini kita akan menggunakan Euclidean distance
sebagai distance measures.

Dalam R, untuk menghitung dan memvisualisasikan distance matrix, kita
dapat menggunakan fungsi ``get\_dist()'' dan ``fviz\_dist()''. Di
visualisasi ini kita akan menggunakan Warna merah untuk menunjukkan
adanya perbedaan dan warna biru mennujukkan adanya persamaan/Kemiripan.

Keterangan Fungsi: - get\_dist: untuk menghitung distance matrix antar
raw. Default: Euclidean Distance - fviz\_dist: untuk memvisualisasi
distance matrix.

\begin{Shaded}
\begin{Highlighting}[]
\CommentTok{#Menhitung Ditance Matrix menggunakan fungsi get_dist()}
\CommentTok{#distance <- _____(dfnorm, method = "euclidean" )}
\NormalTok{distance <-}\StringTok{ }\KeywordTok{get_dist}\NormalTok{(dfnorm, }\DataTypeTok{method =} \StringTok{"euclidean"}\NormalTok{ )}

\CommentTok{#Memvisualisasikan Distance Matrix menggunakan fungsi fviz_dist()}
\CommentTok{#fviz_dist(distance, gradient = list(low = "#00AFBB", mid = "white", high = "#FC4E07"))}
\CommentTok{#tanpa diwarnai, akan ada warna default dari R nya}
\KeywordTok{fviz_dist}\NormalTok{(distance)}
\end{Highlighting}
\end{Shaded}

\includegraphics{1_Clustering_Model_files/figure-latex/unnamed-chunk-6-1.pdf}

dari gambar hasil grafik, melihat similarity dari fearture yang kita
punya

\#Computing k-means clustering in R Kita dapat menghitung k-means dalam
R dengan fungsi ``kmeans''. kali ini kita akan mengelompokkan data
menjadi dua kelompok (centers = 2).

Fungsi kmeans juga memiliki opsi nstart yang mencoba beberapa
konfigurasi awal dan menginformasikan yang terbaik. Contohnya,
menambahkan nilai nstart = 25 akan menghasilkan 25 konfigurasi awal.

\begin{Shaded}
\begin{Highlighting}[]
\CommentTok{#Membuat Model K-Means Klustering dengan Jumlah K/Centers =2, nstart = 25, dengan nama K2}
\NormalTok{k2 <-}\StringTok{ }\KeywordTok{kmeans}\NormalTok{(dfnorm, }\DataTypeTok{centers =} \DecValTok{2}\NormalTok{, }\DataTypeTok{nstart =} \DecValTok{25}\NormalTok{)}
\KeywordTok{str}\NormalTok{(k2)}
\end{Highlighting}
\end{Shaded}

\begin{verbatim}
## List of 9
##  $ cluster     : Named int [1:50] 1 1 1 2 1 1 2 2 1 1 ...
##   ..- attr(*, "names")= chr [1:50] "Alabama" "Alaska" "Arizona" "Arkansas" ...
##  $ centers     : num [1:2, 1:4] 1.005 -0.67 1.014 -0.676 0.198 ...
##   ..- attr(*, "dimnames")=List of 2
##   .. ..$ : chr [1:2] "1" "2"
##   .. ..$ : chr [1:4] "Murder" "Assault" "UrbanPop" "Rape"
##  $ totss       : num 196
##  $ withinss    : num [1:2] 46.7 56.1
##  $ tot.withinss: num 103
##  $ betweenss   : num 93.1
##  $ size        : int [1:2] 20 30
##  $ iter        : int 1
##  $ ifault      : int 0
##  - attr(*, "class")= chr "kmeans"
\end{verbatim}

Keterangan: 1.cluster -\textgreater{} Sebuah vektor bilangan bulat (dari
1: k) yang menunjukkan cluster yang dialokasikan setiap titik.
2.center-\textgreater{} pusat dari matrix cluster. 3.totss
-\textgreater{} total sum of squares. 4.withinss -\textgreater{} Vektor
jumlah dalam-cluster kuadrat, satu komponen per cluster. 5.tot.withinss
-\textgreater{} Total within-cluster sum of squares, i.e.~sum(withinss).
6.betweenss -\textgreater{} The between-cluster sum of squares,
i.e.~\(totss-tot.withinss\). 7.size -\textgreater{} Jumlah Point
disetiap Cluster. yang tergabung di cluster 1 adalah 30, sedangkan
kluster 2 = 20

Kita juga dapat memprint hasil kluster.

\begin{Shaded}
\begin{Highlighting}[]
\CommentTok{#Print Hasil Kluster}
\NormalTok{k2}
\end{Highlighting}
\end{Shaded}

\begin{verbatim}
## K-means clustering with 2 clusters of sizes 20, 30
## 
## Cluster means:
##      Murder    Assault   UrbanPop       Rape
## 1  1.004934  1.0138274  0.1975853  0.8469650
## 2 -0.669956 -0.6758849 -0.1317235 -0.5646433
## 
## Clustering vector:
##        Alabama         Alaska        Arizona       Arkansas     California 
##              1              1              1              2              1 
##       Colorado    Connecticut       Delaware        Florida        Georgia 
##              1              2              2              1              1 
##         Hawaii          Idaho       Illinois        Indiana           Iowa 
##              2              2              1              2              2 
##         Kansas       Kentucky      Louisiana          Maine       Maryland 
##              2              2              1              2              1 
##  Massachusetts       Michigan      Minnesota    Mississippi       Missouri 
##              2              1              2              1              1 
##        Montana       Nebraska         Nevada  New Hampshire     New Jersey 
##              2              2              1              2              2 
##     New Mexico       New York North Carolina   North Dakota           Ohio 
##              1              1              1              2              2 
##       Oklahoma         Oregon   Pennsylvania   Rhode Island South Carolina 
##              2              2              2              2              1 
##   South Dakota      Tennessee          Texas           Utah        Vermont 
##              2              1              1              2              2 
##       Virginia     Washington  West Virginia      Wisconsin        Wyoming 
##              2              2              2              2              2 
## 
## Within cluster sum of squares by cluster:
## [1] 46.74796 56.11445
##  (between_SS / total_SS =  47.5 %)
## 
## Available components:
## 
## [1] "cluster"      "centers"      "totss"        "withinss"    
## [5] "tot.withinss" "betweenss"    "size"         "iter"        
## [9] "ifault"
\end{verbatim}

kita juga dapat melihat hasil dengan menggunakan ``fviz\_cluster''. Ini
memberikan ilustrasi yang bagus tentang cluster.

\begin{Shaded}
\begin{Highlighting}[]
\CommentTok{# visualisasi kluster menggunakan fungsi fviz_cluster()}
\KeywordTok{fviz_cluster}\NormalTok{(k2, }\DataTypeTok{data =}\NormalTok{ dfnorm)}
\end{Highlighting}
\end{Shaded}

\includegraphics{1_Clustering_Model_files/figure-latex/unnamed-chunk-9-1.pdf}

Sebagai alternatif, kita dapat menggunakan scatter plots untuk
menggambarkan kelompok dibandingkan dengan variabel asli.

\begin{Shaded}
\begin{Highlighting}[]
\CommentTok{#Visualisasi ScatterPlots menggunakan ggplot2}
\NormalTok{dfnorm }\OperatorTok
\StringTok{  }\KeywordTok{as_tibble}\NormalTok{() }\OperatorTok
\StringTok{  }\KeywordTok{mutate}\NormalTok{(}\DataTypeTok{cluster =}\NormalTok{ k2}\OperatorTok{$}\NormalTok{cluster,}
         \DataTypeTok{state =} \KeywordTok{row.names}\NormalTok{(USArrests)) }\OperatorTok
\StringTok{  }\KeywordTok{ggplot}\NormalTok{(}\KeywordTok{aes}\NormalTok{(UrbanPop, Murder, }\DataTypeTok{color =} \KeywordTok{factor}\NormalTok{(cluster), }\DataTypeTok{label =}\NormalTok{ state)) }\OperatorTok{+}
\StringTok{  }\KeywordTok{geom_text}\NormalTok{()}
\end{Highlighting}
\end{Shaded}

\includegraphics{1_Clustering_Model_files/figure-latex/unnamed-chunk-10-1.pdf}

Karena jumlah cluster (k) harus ditetapkan sebelum kita memulai
algoritma, seringkali menguntungkan untuk menggunakan beberapa nilai k
yang berbeda dan memeriksa perbedaan dalam hasil. Kita dapat menjalankan
proses yang sama untuk 3, 4, dan 5 cluster.

\begin{Shaded}
\begin{Highlighting}[]
\CommentTok{#Membuat Untuk kluster dengan masing-masing K =3.4.5}
\NormalTok{k3 <-}\StringTok{ }\KeywordTok{kmeans}\NormalTok{(dfnorm, }\DataTypeTok{centers =} \DecValTok{3}\NormalTok{, }\DataTypeTok{nstart =} \DecValTok{25}\NormalTok{)}
\NormalTok{k4 <-}\StringTok{ }\KeywordTok{kmeans}\NormalTok{(dfnorm, }\DataTypeTok{centers =} \DecValTok{4}\NormalTok{, }\DataTypeTok{nstart =} \DecValTok{25}\NormalTok{)}
\NormalTok{k5 <-}\StringTok{ }\KeywordTok{kmeans}\NormalTok{(dfnorm, }\DataTypeTok{centers =} \DecValTok{5}\NormalTok{, }\DataTypeTok{nstart =} \DecValTok{25}\NormalTok{)}

\CommentTok{# Membuat Komparasi Plot}
\NormalTok{p1 <-}\StringTok{ }\KeywordTok{fviz_cluster}\NormalTok{(k2, }\DataTypeTok{geom =} \StringTok{"point"}\NormalTok{, }\DataTypeTok{data =}\NormalTok{ dfnorm) }\OperatorTok{+}\StringTok{ }\KeywordTok{ggtitle}\NormalTok{(}\StringTok{"k = 2"}\NormalTok{)}
\NormalTok{p2 <-}\StringTok{ }\KeywordTok{fviz_cluster}\NormalTok{(k3, }\DataTypeTok{geom =} \StringTok{"point"}\NormalTok{, }\DataTypeTok{data =}\NormalTok{ dfnorm) }\OperatorTok{+}\StringTok{ }\KeywordTok{ggtitle}\NormalTok{(}\StringTok{"k = 3"}\NormalTok{)}
\NormalTok{p3 <-}\StringTok{ }\KeywordTok{fviz_cluster}\NormalTok{(k4, }\DataTypeTok{geom =} \StringTok{"point"}\NormalTok{, }\DataTypeTok{data =}\NormalTok{ dfnorm) }\OperatorTok{+}\StringTok{ }\KeywordTok{ggtitle}\NormalTok{(}\StringTok{"k = 4"}\NormalTok{)}
\NormalTok{p4 <-}\StringTok{ }\KeywordTok{fviz_cluster}\NormalTok{(k5, }\DataTypeTok{geom =} \StringTok{"point"}\NormalTok{, }\DataTypeTok{data =}\NormalTok{ dfnorm) }\OperatorTok{+}\StringTok{ }\KeywordTok{ggtitle}\NormalTok{(}\StringTok{"k = 5"}\NormalTok{)}

\CommentTok{#install packages gridExtra}
\KeywordTok{library}\NormalTok{(gridExtra) }
\end{Highlighting}
\end{Shaded}

\begin{verbatim}
## 
## Attaching package: 'gridExtra'
\end{verbatim}

\begin{verbatim}
## The following object is masked from 'package:dplyr':
## 
##     combine
\end{verbatim}

\begin{Shaded}
\begin{Highlighting}[]
\KeywordTok{grid.arrange}\NormalTok{(p1, p2, p3, p4, }\DataTypeTok{nrow =} \DecValTok{2}\NormalTok{)}
\end{Highlighting}
\end{Shaded}

\includegraphics{1_Clustering_Model_files/figure-latex/unnamed-chunk-11-1.pdf}

\#\#Determining Optimal Clusters Untuk membantu analis, berikut ini
merupakan dua metode yang populer untuk menentukan nilai Optimal CLuter:
1. Elbow method 2. Silhouette method

\#Elbow Method

\begin{Shaded}
\begin{Highlighting}[]
\KeywordTok{set.seed}\NormalTok{(}\DecValTok{123}\NormalTok{)}

\CommentTok{# function to compute total within-cluster sum of square }
\NormalTok{wss <-}\StringTok{ }\ControlFlowTok{function}\NormalTok{(k) \{}
  \KeywordTok{kmeans}\NormalTok{(dfnorm, k, }\DataTypeTok{nstart =} \DecValTok{10}\NormalTok{ )}\OperatorTok{$}\NormalTok{tot.withinss}
\NormalTok{\}}

\CommentTok{# Compute and plot wss for k = 1 to k = 15}
\NormalTok{k.values <-}\StringTok{ }\DecValTok{1}\OperatorTok{:}\DecValTok{15}

\CommentTok{# extract wss for 2-15 clusters}
\NormalTok{wss_values <-}\StringTok{ }\KeywordTok{map_dbl}\NormalTok{(k.values, wss)}

\KeywordTok{plot}\NormalTok{(k.values, wss_values,}
       \DataTypeTok{type=}\StringTok{"b"}\NormalTok{, }\DataTypeTok{pch =} \DecValTok{19}\NormalTok{, }\DataTypeTok{frame =} \OtherTok{FALSE}\NormalTok{, }
       \DataTypeTok{xlab=}\StringTok{"Number of clusters K"}\NormalTok{,}
       \DataTypeTok{ylab=}\StringTok{"Total within-clusters sum of squares"}\NormalTok{)}
\end{Highlighting}
\end{Shaded}

\includegraphics{1_Clustering_Model_files/figure-latex/unnamed-chunk-12-1.pdf}
Hasilnya menunjukkan bahwa 4 adalah jumlah cluster optimal karena berada
diposisi siku. TItik tengah dari siku. siku = 23456, maka titik tengah =
4

Kita dengan cepat dapat menghitung Elbow Method menggunakan fungsi
fviz\_nbclust()

\begin{Shaded}
\begin{Highlighting}[]
\CommentTok{#Visualisasi Elbow Method menggunakan fungsi fviz_nbclust()}
\KeywordTok{fviz_nbclust}\NormalTok{(dfnorm, kmeans, }\DataTypeTok{method =} \StringTok{"wss"}\NormalTok{)}
\end{Highlighting}
\end{Shaded}

\includegraphics{1_Clustering_Model_files/figure-latex/unnamed-chunk-13-1.pdf}

\#Average Silhouette Method Singkatnya, pendekatan siluet mengukur
kualitas pengelompokan. Artinya, menentukan seberapa baik setiap objek
terletak di dalam klusternya. Lebar siluet rata-rata yang tinggi
menunjukkan pengelompokan yang baik

Kita bisa menggunakan fungsi Silhouette dalam package cluster untuk
menghitung lebar Silhouette rata-rata. code berikut menghitung
pendekatan ini untuk 1-15 cluster. Hasilnya menunjukkan bahwa 2 klaster
memaksimalkan nilai rata-rata Silhouette dengan 4 klaster yang masuk
sebagai jumlah klaster optimal kedua.

\begin{Shaded}
\begin{Highlighting}[]
\CommentTok{# function to compute average silhouette for k clusters }
\NormalTok{avg_sil <-}\StringTok{ }\ControlFlowTok{function}\NormalTok{(k) \{}
\NormalTok{  km.res <-}\StringTok{ }\KeywordTok{kmeans}\NormalTok{(dfnorm, }\DataTypeTok{centers =}\NormalTok{ k, }\DataTypeTok{nstart =} \DecValTok{25}\NormalTok{)}
\NormalTok{  ss <-}\StringTok{ }\KeywordTok{silhouette}\NormalTok{(km.res}\OperatorTok{$}\NormalTok{cluster, }\KeywordTok{dist}\NormalTok{(df))}
  \KeywordTok{mean}\NormalTok{(ss[, }\DecValTok{3}\NormalTok{])}
\NormalTok{\}}

\CommentTok{# Compute and plot wss for k = 2 to k = 15}
\NormalTok{k.values <-}\StringTok{ }\DecValTok{2}\OperatorTok{:}\DecValTok{15}

\CommentTok{# extract avg silhouette for 2-15 clusters}
\NormalTok{avg_sil_values <-}\StringTok{ }\KeywordTok{map_dbl}\NormalTok{(k.values, avg_sil)}

\KeywordTok{plot}\NormalTok{(k.values, avg_sil_values,}
       \DataTypeTok{type =} \StringTok{"b"}\NormalTok{, }\DataTypeTok{pch =} \DecValTok{19}\NormalTok{, }\DataTypeTok{frame =} \OtherTok{FALSE}\NormalTok{, }
       \DataTypeTok{xlab =} \StringTok{"Number of clusters K"}\NormalTok{,}
       \DataTypeTok{ylab =} \StringTok{"Average Silhouettes"}\NormalTok{)}
\end{Highlighting}
\end{Shaded}

\includegraphics{1_Clustering_Model_files/figure-latex/unnamed-chunk-14-1.pdf}
paling bagus cluster paling tinggi di 2.

Kita dengan cepat dapat menghitung Average Silhouette dalam satu fungsi
fviz\_nbclust.

\begin{Shaded}
\begin{Highlighting}[]
\CommentTok{#Visualisasi Average Silhoutte menggunakan fugsi fviz_nbclust()}
\KeywordTok{fviz_nbclust}\NormalTok{(dfnorm, kmeans, }\DataTypeTok{method =} \StringTok{"silhouette"}\NormalTok{)}
\end{Highlighting}
\end{Shaded}

\includegraphics{1_Clustering_Model_files/figure-latex/unnamed-chunk-15-1.pdf}
paling bagus cluster paling tinggi di 2.

\#Extracting Results Dengan sebagian besar pendekatan ini menyarankan 4
sebagai jumlah cluster optimal, kita dapat melakukan analisis akhir dan
mengekstraksi hasilnya menggunakan 4 cluster.

\begin{Shaded}
\begin{Highlighting}[]
\CommentTok{# Compute k-means clustering with k = 4}
\KeywordTok{set.seed}\NormalTok{(}\DecValTok{123}\NormalTok{)}
\NormalTok{final <-}\StringTok{ }\KeywordTok{kmeans}\NormalTok{(dfnorm, }\DecValTok{4}\NormalTok{, }\DataTypeTok{nstart =} \DecValTok{25}\NormalTok{)}
\KeywordTok{print}\NormalTok{(final)}
\end{Highlighting}
\end{Shaded}

\begin{verbatim}
## K-means clustering with 4 clusters of sizes 8, 13, 16, 13
## 
## Cluster means:
##       Murder    Assault   UrbanPop        Rape
## 1  1.4118898  0.8743346 -0.8145211  0.01927104
## 2 -0.9615407 -1.1066010 -0.9301069 -0.96676331
## 3 -0.4894375 -0.3826001  0.5758298 -0.26165379
## 4  0.6950701  1.0394414  0.7226370  1.27693964
## 
## Clustering vector:
##        Alabama         Alaska        Arizona       Arkansas     California 
##              1              4              4              1              4 
##       Colorado    Connecticut       Delaware        Florida        Georgia 
##              4              3              3              4              1 
##         Hawaii          Idaho       Illinois        Indiana           Iowa 
##              3              2              4              3              2 
##         Kansas       Kentucky      Louisiana          Maine       Maryland 
##              3              2              1              2              4 
##  Massachusetts       Michigan      Minnesota    Mississippi       Missouri 
##              3              4              2              1              4 
##        Montana       Nebraska         Nevada  New Hampshire     New Jersey 
##              2              2              4              2              3 
##     New Mexico       New York North Carolina   North Dakota           Ohio 
##              4              4              1              2              3 
##       Oklahoma         Oregon   Pennsylvania   Rhode Island South Carolina 
##              3              3              3              3              1 
##   South Dakota      Tennessee          Texas           Utah        Vermont 
##              2              1              4              3              2 
##       Virginia     Washington  West Virginia      Wisconsin        Wyoming 
##              3              3              2              2              3 
## 
## Within cluster sum of squares by cluster:
## [1]  8.316061 11.952463 16.212213 19.922437
##  (between_SS / total_SS =  71.2 %)
## 
## Available components:
## 
## [1] "cluster"      "centers"      "totss"        "withinss"    
## [5] "tot.withinss" "betweenss"    "size"         "iter"        
## [9] "ifault"
\end{verbatim}

kita dapat memvisualisasikan hasil menggunakan (fviz\_cluster)

\begin{Shaded}
\begin{Highlighting}[]
\CommentTok{# Visualisasi Final Cluster}
\KeywordTok{fviz_cluster}\NormalTok{(final, }\DataTypeTok{data =}\NormalTok{ dfnorm)}
\end{Highlighting}
\end{Shaded}

\includegraphics{1_Clustering_Model_files/figure-latex/unnamed-chunk-17-1.pdf}

Dan kita dapat mengekstrak cluster dan menambah data awal kita untuk
melakukan beberapa statistik deskriptif (rata-rata) di tingkat cluster

\begin{Shaded}
\begin{Highlighting}[]
\NormalTok{USArrests }\OperatorTok
\StringTok{  }\KeywordTok{mutate}\NormalTok{(}\DataTypeTok{Cluster =}\NormalTok{ final}\OperatorTok{$}\NormalTok{cluster) }\OperatorTok
\StringTok{  }\KeywordTok{group_by}\NormalTok{(Cluster) }\OperatorTok
\StringTok{  }\KeywordTok{summarise_all}\NormalTok{(}\StringTok{"mean"}\NormalTok{)}
\end{Highlighting}
\end{Shaded}

\begin{verbatim}
## # A tibble: 4 x 5
##   Cluster Murder Assault UrbanPop  Rape
##     <int>  <dbl>   <dbl>    <dbl> <dbl>
## 1       1  13.9    244.      53.8  21.4
## 2       2   3.6     78.5     52.1  12.2
## 3       3   5.66   139.      73.9  18.8
## 4       4  10.8    257.      76    33.2
\end{verbatim}


\end{document}
